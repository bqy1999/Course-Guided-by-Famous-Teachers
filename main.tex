\documentclass[12pt]{ctexart}
\usepackage[english]{babel}
\usepackage{natbib}
\usepackage{url}
\usepackage[utf8x]{inputenc}
\usepackage{amsmath}
\usepackage{graphicx}
\graphicspath{{images/}}
\usepackage{parskip}
\usepackage{fancyhdr}
\usepackage{vmargin}
\usepackage{indentfirst}
\setmarginsrb{3 cm}{2.5 cm}{3 cm}{2.5 cm}{1 cm}{1.5 cm}{1 cm}{1.5 cm}
\setlength{\parindent}{2em}

\title{人工智能的前世今生和未来}								% Title
\author{毕秋宇}								% Author
\date{\today}											% Date

\makeatletter
\let\thetitle\@title
\let\theauthor\@author
\let\thedate\@date
\makeatother

\pagestyle{fancy}
\fancyhf{}
\rhead{\theauthor}
\lhead{\thetitle}
\cfoot{\thepage}

\begin{document}

%%%%%%%%%%%%%%%%%%%%%%%%%%%%%%%%%%%%%%%%%%%%%%%%%%%%%%%%%%%%%%%%%%%%%%%%%%%%%%%%%%%%%%%%%

\begin{titlepage}
	\centering
    \vspace*{0.5 cm}
    \includegraphics[scale = 0.2]{NJU.jpg}\\[1.0 cm]	% University Logo
    \textsc{\LARGE Nanjing University}\\[2.0 cm]	% University Name
    \textsc{\large 24000010}\\[0.5 cm]				% Course Code
	\textsc{\large Course Guided by Famous Teachers}\\[0.5 cm]				% Course Name
	\rule{\linewidth}{0.2 mm} \\[0.4 cm]
    {\huge \bfseries \thetitle}\\
	\rule{\linewidth}{0.2 mm} \\[1.5 cm]
	
	\begin{minipage}{0.4\textwidth}
		\begin{flushleft} \large
			\emph{Author:}\\
			\theauthor
			\end{flushleft}
			\end{minipage}
			\begin{minipage}{0.4\textwidth}
			\begin{flushright} \large
                \emph{Student Number:} \\
			171860624\\									% Your Student Number
                \emph{Department:} \\
                \emph{Computer Science}
		\end{flushright}
	\end{minipage}\\[2 cm]
	
	{\large \thedate}\\[2 cm]
 
	\vfill
	
\end{titlepage}

%%%%%%%%%%%%%%%%%%%%%%%%%%%%%%%%%%%%%%%%%%%%%%%%%%%%%%%%%%%%%%%%%%%%%%%%%%%%%%%%%%%%%%%%%

\tableofcontents
\pagebreak

%%%%%%%%%%%%%%%%%%%%%%%%%%%%%%%%%%%%%%%%%%%%%%%%%%%%%%%%%%%%%%%%%%%%%%%%%%%%%%%%%%%%%%%%%

\section{摘要}
人工智能,一个近些年突然进入大众视野的名词,现已经成为了政、学、研、投、产等各界人士谈论的最热门话题。
甚至有学者认为其可与前三次工业和科技革命相媲美,足见其将对人类社会带来何等重要的影响。本文就试图以一个计算机科学与技术系在读学生的角度来回顾一下人工智能的发展历程。\\
$\textbf{关键词}$:人工智能;图灵;发展历程;未来

\section{人工智能起源}
\indent 自1951 年,人工智能之父英国人阿兰 · 图灵一篇里程碑式的论文``Can digital computers think?''\cite{name3}为人类带来了一个新学科—人工智能,已经过了约七十年。
在之后的岁月中,为了证明机器是否能够思考,图灵又发明了 “图灵测试”(Turing Test)\cite{name2} \cite{name4}。即在两个封闭的空间中当人与对面进行对话时,如果有百分之七十的人类分辨不出对面是一台机器还是一个人在与自身对话,就说明机器具有了人的智能。
虽然在未来很长的一段时间内,图灵测试的科学性饱受争议,但是毫无疑问图灵为机器智能的起源和发展做出了极大的贡献。直到现在,计算机学科的最高奖项依然是以图灵命名的图灵奖,一代又一代的计算机人从此走上了研究机器智能的路途。\\
\indent 图灵测试刚刚诞生的时候,人们对于思考和智能的理解还不够深入,这也注定了人工智能的发展之路不会一帆风顺,1980 年约翰 · 塞尔在``Minds, brains, and programs''\cite{name5}一文中构造了一个中文屋子思想实验。在该实验中,约翰 · 塞尔想象他被锁在一间屋子里,手动模拟一个巨大的人工智能程序,和外界进行中文的对话。这个程序据说是 “懂中文” 的——至少,能以中文通过图灵测试。
但是屋子里除了一堆纸(塞尔先生的原话是“bits of paper”)上写着运算的规则之外,别无他物。
塞尔不懂中文,在屋子里摆弄符号显然也无助于他习得中文,屋子里也没有别的东西理解中文了。如果塞尔不 “理解” 中文,那么塞尔加上这堆纸也不能说是 “理解” 中文。那么虽然屋子的中文水平足以骗过中文使用者,但没有任何实体真的 “理解” 发生了什么。
换言之,所谓的图灵测试也是没有用的,就算通过了它也不能表明计算机在思考。\\
\indent 但是在我看来,虽然图灵测试的科学性可能没有民间普遍认为的那么高,那么它真正的意义在于激发了人们对于人工智能的兴趣,为当时生产力和科技水平并不发达的人们去探寻人工智能的奥秘提供了一条切实可行的路径。
在计算机世界有很多的俗语来形容这样的事,例如``忍住优化的冲动'',``先实现再优化''等,就类似图灵测试,虽然它对于智能的定义并不是那么的准确,但是能够引领人们去尝试实现一个能够达成图灵测试的新机器,在我看来也是无量的贡献。
当然,现如今也还存在有许多图灵测试的忠实支持者,例如Douglas Hofstadter先生等人,认为如今的整个人工智能领域都背离了初衷,这也是另外一种对于图灵测试的解读了。\\
\indent 1956年 “人工智能” 首次在达特茅斯会议中被提出,John McCarthy, Marvin Minsky, Allen Newell, Arthur Samuel 以及 Herbert Simon 五人顺势成为当时这一领域的领军人物。紧接着人工智能开始酝酿其第一次浪潮,人工智能实验室在全球各地扎根。在那个人工智能的风口浪尖上,人工智能五大领袖们开始对这个领域的发展前景表现出极其的乐观,史称人工智能第一波高潮。
而正如之前所言,人工智能的发展必然不是如此的一帆风顺。虽然政府以及民间科研机构对于人工智能投入了极大的资金和关注,但是当时是1956年,据世界上第一台计算机``ENIAC''诞生仅仅十年,计算机的体积以及计算能力远没有到达如今的地步。如此多的注视和投入,恰如让一个刚刚蹒跚学步的孩童去学习大学的知识,必是揠苗助长、难上加难。
在达特茅斯会议之后相当长的一段时间内,人工智能领域未作出任何对于世界有重大影响力的成就,人们的热情也随着长久未有产出而日渐消退。
这种情况一直持续到了1973 年,以《莱特希尔报告》的推出为代表,象征着人工智能正式进入寒冬。这篇报告宣称``人工智能领域的任何一部分都没有能产出人们当初承诺的有主要影响力进步''。各国政府勒令大规模削减人工智能方面的投入。这之后的十年间,人工智能鲜有被人提起。
% \newpage

\section{寒冬中求索}
\indent 1973年进入第一次人工智能寒冬除了上述的硬件原因外还存在其他更为现实的原因,当时的人工智能由于只能解决Toy domain而饱受批评。那时的人工智能语音只能做到大约十个单词,机器视觉甚至还不能够分辨出人和椅子,由于移动互联网还没有起步更不能对个人产生什么可以察觉到的影响。
另外一个不可忽视的原因就是1956年美国开启了越南战争,在随后的战争泥潭还有石油危机中,美国消耗了大量的国家实力连带着经济也遭受到了冲击。当时国际上的政客对于国家在人工智能领域的投入普遍持消极态度,英国著名的学者Lighthill甚至直言人工智能就是在浪费钱。
人工智能的研究经费也因此遭到大幅削减,随后英国的政府还停止了对三所大学Edinburgh,Sussex和Essex的人工智能研究资助。\\
\indent 即使是在如此艰难的情况下,计算机人也没有放弃对于人工智能的探索,在那段人工智能的寒冬中,人们加强了对于语言神经等多方面的认识,
许多在将来有极大影响力的工作,例如AK Mackworth对于关系网络的研究论文``Consistency in networks of relations''\cite{name6}就是在这段时间发表的。没有了民间的殷切期待,科学家们反而更能够从兴趣出发来完成科研工作。将来赫赫有名的机器学习算法,很多就来源于那个时期人们的某一道思想闪光。
虽说这些思想闪光在当时无法实现,但是在几十年之后的今天,在硬件软件水平和数据规模有了极大提升的今天,这些思想闪光就能够真正地从象牙塔中走了出来,为如今人工智能的发展提供不竭的动力!
% \newpage

\section{曲折又回还}
\indent 到了上世纪八十年代,由于世界局势趋于缓和,政府和民间对于人工智能的关注又多了起来。当时,由于专家系统的崛起而使得人工智能再次迎来一次久旱之后的甘霖期,也是一次新的高潮。
所谓专家系统是人工智能的一个重要的分支,专家系统可以看作是一类具有专门知识和经验的计算机智能程序系统,简而言之就是采用七十年代发展起来的人工智能中的知识表示和知识推理技术来模拟通常由领域专家才能解决的复杂问题。
与以往的Toy domain有所不同,这时期的人工智能不在局限于解决一些小儿科的问题,而是被用于完成那些没有公认的理论和方法、数据不精确或信息不完整、人类专家短缺或专门知识十分昂贵的诊断、解释、监控、预测、规划和设计等任务。
套用政治学的术语来说,专家系统执行的任务或者说解决的问题主要是知识密集型的。也正因为如此,专家系统能为人们带来明显的经济效益,有了它执行许多任务是不用再费时费力地去寻找相应方向的专业人士,这样可以极大地节省开支。
在这一时期,南京大学也有优秀的很多研究学者投入了专家系统的开发中,甚至是在今天,南京大学在专家系统这一领域仍然有一定的话语权。\\
\indent 但对于多灾多难的人工智能领域来说,快乐的时光总是短暂的。人们随后就意识到了人工智能的问题不仅仅是硬件的问题,人工智能的发展缓慢更多是因为软件以及算法层面的挑战没有被突破。
在机器学习和人工智能算法迟迟没有进展的情况下,硬件也遭遇危机。随着1987年基于通用计算的Lisp机器在商业上的失败,人工智能的研究又一次滑入了低迷期。有了之前一次低迷期的经验,这一次人们对于未来的发展有了更加大的信心。
如今人工智能领域应用最为广泛的算法(甚至Alpha Go \cite{name7}也是基于其开发的)——深度学习算法体系,正是在这一事件建立的。包括蒙特卡洛搜索书、深层神经网络、数学卷积在计算机视觉中的应用、特征理论、Softmax 激活函数在内的等等等等数不胜数的优秀算法在这一时间被提出。
科研工作者们对于寒冬的结束抱有极大的信心,所需要等待的只是一阵东风——一个恰当的机会罢了。\\
% \newpage

\section{今日与未来}
\indent 果不其然,再一次的高潮很快便来临了,并且一直持续到了如今。随着计算机计算能力的不断提高,以数据挖掘和商业诊断为主要代表的人工智能的应用非常成功,使其得以重回人们的视野。
国际象棋、日本将棋、中国围棋,人工智能在棋类运动中的披荆斩棘使得普罗大众也能够认识到人工智能的魅力,也正因为如此,人工智能再一次引起了全球各个政府部门的关注。
先是政府部门对于安保的投资促使了计算机视觉的巨大进步,就拿中国举例,现如今中国的几十万公安摄像头,真正由人来操控的已经是少之又少了。大多数是人工智能在背后支撑,计算机视觉已经发展到了常人所难以理解的地步,数据库中的网络逃犯只要出现在公安摄像头之下就有极大的可能被人工智能识别出来,这为社会的稳定起到了巨大作用。
当然这还存在着许多的不足,例如在摄像头的分辨率不够的情况下,人工智能分辨逃犯的准确度就会受到较大的影响,这也可以说是将来计算机视觉的一大发展方向,即模糊的影响,算法鲁棒性的评定。
如今人工智能技术已广泛应用于智能驾驶、智慧安防、智慧金融、智慧零售等新的行业形态,并为人们的生活带来了很大的便利,可以说人工智能真正拥有了``智'',对得起它的这个名号了。\\
\indent 在十几年迅猛的发展之后,最近人工智能似乎有遇到了一些前进的阻力,于是有人开始担心是否是第三次人工智能寒冬即将到来。我个人观点认为,这种担忧是没有太多的必要的。
目前人工智能领域的领军人物吴恩达、李飞飞以及南京大学的周志华教师等人也给出了积极的看法。吴恩达在某次媒体采访中也表示``他认为,人工智能不会再度经历一个冬天。之前的人工智能冬天到来,是因为人工智能并没有创造太多的经济价值,而且相对来说,人工智能研究团队的工作被夸大了。在今天来看,人工智能已经成为一种类似于互联网和电力的通用技术,能够适用于许多行业。人工智能的兴起已经有了坚实的基础,有大量的公司正在通过人工智能来获取收入,人工智能已经有了一个清晰的路线图来创造大量的价值。自动化开始变得无处不在,这是不会消失的。''
这可以说是非常中肯的评价了,但这并不意味的我们可以盲目的乐观,在科学史上对于某一个方向的盲目乐观已经给我们带来了太多太多的教训。所以未来固然是光明的,人工智能也固然是会持续发展的,但这仍需要全世界顶尖的头脑参与其中,为这一学科的发展发光发热。\\

\section{总结}
\indent 人工智能发展至今经历了太多的风雨,其中滋味知者自知,预测它的走向就如预测股票一样难以捉摸,胡乱的猜测只能够是杞人忧天。最后愿以一句名言来结束此文,人工智能的未来算是``革命尚未成功,同志仍需努力''。
\newpage

\bibliographystyle{plain}
\bibliography{biblist}

\end{document}
